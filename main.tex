\documentclass{IEEEtran}

\usepackage{amsmath}
\usepackage{amssymb}
\usepackage{amsfonts}

\ifCLASSINFOpdf
   \usepackage[pdftex]{graphicx}
\else
   \usepackage[dvips]{graphicx}
\fi

\ifCLASSOPTIONcompsoc
  \usepackage[caption=false, font=normalsize, labelfont=sf, textfont=sf]{subfig}
\else
  \usepackage[caption=false, font=footnotesize]{subfig}
\fi

\usepackage{textcomp}
\usepackage{nicefrac}
\usepackage{siunitx}
\usepackage{fancyref}

\usepackage[style=ieee, doi=false, isbn=false, url=false, maxbibnames=1, minbibnames=1, maxcitenames=1, mincitenames=1, backend=biber, defernumbers=false]{biblatex}
\addbibresource{./references.bib}

% \AtEveryBibitem{\clearfield{month}}
% \AtEveryBibitem{\clearfield{day}}
% \AtEveryBibitem{\clearfield{volume}}
% \AtEveryBibitem{\clearfield{issue}}
% \AtEveryBibitem{\clearfield{pages}}
% \AtEveryBibitem{\clearfield{number}}
% \AtEveryBibitem{\clearfield{title}}
% \AtEveryBibitem{\clearfield{isbn}}
% \AtEveryBibitem{\clearfield{keywords}}
% \AtEveryBibitem{\clearfield{issn}}
% \AtEveryBibitem{\clearfield{journal}}

\usepackage[activate={true, nocompatibility}, final, tracking=true, kerning=true, spacing=true, factor=1100, stretch=10, shrink=10]{microtype}
% \linespread{0.9}

\usepackage{glossaries}

\glsaddkey* {longs}
{\glsentrylong{\glslabel}s}
{\glsentrylongs}
{\Glsentrylongs}
{\glslongs}
{\Glslongs}
{\GLSlongs}

\glsaddkey* {shorts}
{\glsentryshort{\glslabel}s}
{\glsentryshorts}
{\Glsentryshorts}
{\glsshorts}
{\Glsshorts}
{\GLSshorts}

\DeclareRobustCommand{\glss}[1]{
  \ifglsused{#1}{\glsshorts{#1}}{\glslongs{#1} (\glsshorts{#1})\glsunset{#1}}
}

\DeclareRobustCommand{\Glss}[1]{
  \ifglsused{#1}{\Glsshorts{#1}}{\Glslongs{#1} (\glsshorts{#1})\glsunset{#1}}
}

\newacronym{CNN}{CNN}{Convolutional Neural Network}
\newacronym{CT}{CT}{Computed Tomography}
\newacronym{CV}{CV}{Cross Validation}
\newacronym{FOV}{FOV}{Field of View}
\newacronym[longs={Interstitial Lung Disease}, shorts={ILDs}]{ILD}{ILD}{Interstitial Lung Disease}
\newacronym{IPF}{IPF}{Idiopathic Pulmonary Fibrosis}
\newacronym{MB}{MB}{Memory Bank}
\newacronym{MAE}{MAE}{Mean Absolute Error}
\newacronym{NN}{NN}{Neural Network}
\newacronym{OSIC}{OSIC}{Open Source Imaging Consortium}
\newacronym{PReLU}{PReLU}{Parametric Rectified Linear Unit}
\newacronym{RAE}{RAE}{Relative Absolute Error}
\newacronym{STD}{STD}{Standard Deviation}


\begin{document}
    \title{
        % \vspace{-0.75cm}
        
        Survival Analysis of Idiopathic Pulmonary Fibrosis Patients from CT
    }
    
    \author{
        % \vspace{-0.25cm}

        Ahmed~H.~Shahin,
        Alexander~C.~Whitehead~\IEEEmembership{Student~Member,~IEEE},
        An~Zhao,
        Daniel~C.~Alexander,
        Joseph~Jacob,
        and~David~Barber
    
        % \vspace{-0.75cm}

        \thanks{
            % \scriptsize
            This work was funded by \gls{OSIC}.
        }
        \thanks{
            % \scriptsize
            Ahmed~H.~Shahin, Alexander~C.~Whitehead, An~Zhao, Daniel~C.~Alexander, and David~Barber are with the Department of Computer Science, University College London, London, UK.
        }
        \thanks{
            % \scriptsize
            Ahmed~H.~Shahin, An~Zhao, Daniel~C.~Alexander, and Joseph~Jacob are with the Centre for Medical Image Computing, University College London, London, UK.
        }
        \thanks{
            % \scriptsize
            Joseph~Jacob is with Lungs for Living Research Centre, University College London, London, UK.
        }
        \thanks{
            % \scriptsize
            David~Barber is with the Centre for Artificial Intelligence, University College London, London, UK.
        }
        \thanks{
            % \scriptsize
            (contact: \texttt{alexander.whitehead.18@ucl.ac.uk}).
        }
    }
    
    \pagestyle{plain}
    \pagenumbering{gobble}
    
    \maketitle
    
    \begin{abstract}
        Idiopathic pulmonary fibrosis is a deadly disease with a non-linear heterogeneous progression. It would be useful both clinically, and for research, to be able to predict how this disease will progress, right up to the length of time for which a patient is expected to survive. Survival analysis is one such way to do this. Traditional methods, such as Cox, seek to rank participants based on their predicted survivability. However, Cox cannot directly output a survival time. DeepHit is a neural network based approach which simplifies to a classification problem, to predict the most likely histogram bin of survival time. A disadvantage of DeepHit is that an error of one year is equivalent to an error of one hundred years. A common problem encountered with survival models is that training data is often censored, where the exact time of death is unknown except that it is past a censoring time. Here, a comparison of neural network approaches utilising five different losses is presented. Three base losses are used, likelihood, Cox, and DeepHit. Two variations of likelihood (one which samples censoring time from a uniform and one from a Gaussian distribution) and two variations of Cox (classical and where a memory bank of previous predictions is used during ranking) are used. The input to each model is a single computed tomography volume (plus optionally information regarding the patients age and sex etc) and the output is a survival time. Improvements over previous work includes; a larger model with a learned downsampling, a parameterised activation (which starts linear and becomes non-linear), a softplus output, orthogonal initialisation, an optimiser integrating weight decay, gradient accumulation, and an annealed learning rate. Evaluations used include; mean and relative absolute error, the concordance index, the Brier score, and a visual analysis of Grad-CAM results. Overall, the likelihood trained models performed the best, with DeepHit a close second and Cox a distant last. The model trained with a likelihood using a uniform sampling of censoring time performed marginally better than the alternative. Results seemed mostly inconclusive about the incorporation of clinical features.
    \end{abstract}
    
    % \begin{IEEEkeywords}
        
    % \end{IEEEkeywords}
    
    \vspace{-0.75cm}

\section{Introduction} \label{sec:introduction}
    \IEEEPARstart{U}{nder} the umbrella of \glss{ILD}, \gls{IPF} is  characterised by a buildup of scar tissue in and a stiffening of the lungs of the patient; leading to a reduction in the volume of the lung, resulting in a shortness of breath and eventually death. As with many other \glss{ILD}, the progression of the disease is heterogeneous, and prognosis is challenging~\cite{King2011IdiopathicFibrosis}.
    
    Previously, methods to monitor \gls{IPF} included testing lung function, with spirometric measurement of lung volume~\cite{Watters1986AFibrosis}, or taking \gls{CT} acquisitions over time~\cite{Lynch2018DiagnosticPaper}. Both approaches are limited by several factors, including, physical limitations of the patient, technical accuracy (spirometer baseline drift bias), and  longitudinal data availability. In contrast, measuring how long an \gls{IPF} patient could survive on the basis of a baseline \gls{CT} scan can be a useful clinical outcome to measure in order to prioritise resources and plan intervention.% Furthermore, for research, it would be useful to accurately predict survival time in order to perform analysis of the disease itself.
    For this reason, here the focus is on accurately predicting \gls{IPF} patient death time based on a baseline \gls{CT} scan and associated clinical features.

    In Cox Proportional Hazards Survival Analysis~\cite{Cox1972RegressionLife-Tables} the death time of one participant is compared to another, and the model ranks patients according to their expected death times. However, a limitation is that the model does not directly output survival times. Recently models have been introduced that attempt to predict more directly the death time of a participant. For instance, DeepHit uses a \gls{CNN} to perform feature extraction on input \gls{CT} and outputs the probability of a survival time falling within predetermined bins of a death time histogram~\cite{Lee2018DeepHit:Risks}. This treats survival analysis as a multi-class classification problem with a class for each time-bin that a patient could die in. A disadvantage is that the bins are not ordinally related and the model is penalised as much for making an error of one month as it is an error of ten years.

    Censoring is a significant issue in survival analysis in which the precise death time of a patient is unknown. For example, it may be known only that the patient is still alive, or that the patient died only within a specific time interval. In the \gls{OSIC} \gls{IPF} data set~\cite{OSICOSICRepository}, approximately \SI{66}{\percent} of the records are right-censored, meaning that the time of death is above a known value but it is unknown by how much. A simple approach would be to remove censored data, however, this would discard a very significant fraction of training data. Missing data in clinical records is a related issue. Again, with approximately \SI{66}{\percent} of the \gls{OSIC} \gls{IPF} data~\cite{OSICOSICRepository} set has some missing clinical information.

    Here, a number of survival analysis models that predict death time using a \gls{NN} are presented. The inputs of which being a baseline \gls{CT} scan and associated patient clinical information (such as height, age, etc). The models are able to address censoring and missing clinical information following~\cite{Shahin2023DeepAnalysis, Shahin2022SurvivalData} respectively. Different training losses are used, including ones based on classical Cox based ranking, likelihood, and DeepHit. In the case of the Cox based loss, one with and one without a memory bank of previous predictions is used (to allow the loss to be approximated at all previously seen data points~\cite{Shahin2022SurvivalData}). In the case of the likelihood based loss, one where censoring time is sampled in the classical way and one where censoring time is sampled from a uniform distribution is used~\cite{Shahin2023DeepAnalysis}. Extensions over previous work include; a change to the \gls{NN} architecture (larger, with a learnt downsampling, parameterised activation and softplus output, and orthogonal initialisation), a new optimiser, gradient accumulation (as such an increased batch size), and an annealed learning rate.

    % \vspace{-0.5cm}

\section{Methods} \label{sec:methods}
    \subsection{Data Acquisition and Preparation} \label{sec:data_acquisition_and_preparation}
        $550$ \gls{CT} acquisitions were taken from the \gls{OSIC} data set. Each acquisition had a \gls{FOV} containing the base of the lung, heart and upper lung. Each acquisition was segmented to remove data outside of the lung and normalised independently. Where appropriate clinical features, such as age, sex, and medical data. If missing clinical features were present their value was inferred~\cite{Shahin2022SurvivalData}. Data were split into train and test groups following five fold \gls{CV}.

    \subsection{Training} \label{sec:training}
        Each \gls{NN} consisted of seven \gls{CNN} blocks, within which was two convolutions with stride one and one with stride two (to reduce the spatial dimension). Each convolution had a kernel size of three and used an orthogonal activation. Between each layer was a \gls{PReLU} activation~\cite{He2015DelvingClassification}, initialised with an alpha of one, which starts the network linear and allows it to become more non-linear as training progresses. Seven \gls{CNN} blocks were selected as this is the maximum number of downsampling steps possible with the data used. At each downsampling step the number of channels doubled.

        After the feature extraction section a global average pooling and flattening layer were used before being passed through a number of fully connected layers. When clinical features were used, a number of fully connected layers were used to increase the number of units to be equal to the number from the previous step, where the number of units doubled at each layer. At each layer the number of units halved and the number of layers was defined by the number needed to reduce the number of units to the output size. A softplus activation was used at the output for numerical stability.

        AdamW was used as the optimiser with weight decay to improve the convergence rate as well as to penalise against large weights and overfitting~\cite{Loshchilov2019DecoupledRegularization}. The learning rate started at zero and increased linearly to the target learning rate over the first one tenth of iterations before reducing back to zero over the next nine tenths.

        A batch size of four was used (because of Cox loss) but was increased to $32$ using gradient accumulation.

        Five loss functions were used:

        \begin{itemize}
            \item \textbf{Uniform Likelihood} - Gaussian likelihood loss using a uniform distribution to sample censoring time (with a fixed \gls{STD} equal to one year).

            \item \textbf{Likelihood} - Gaussian likelihood loss (with a fixed \gls{STD} equal to one year).

            \item \textbf{Cox} - Cox Proportional Hazards loss (converted to survival times using the Breslow estimator)~\cite{Cox1972RegressionLife-Tables}.

            \item \textbf{Cox \gls{MB}} - Cox Proportional Hazards loss with \gls{MB} (converted to survival times using the Breslow estimator~\cite{Breslow1974CovarianceData})~\cite{Shahin2022SurvivalData}.

            \item \textbf{DeepHit} - DeepHit loss (log-likelihood, with a maximum output value of $105$ years and $840$ bins)~\cite{Lee2018DeepHit:Risks}.
        \end{itemize}

    \subsection{Evaluation} \label{sec:evaluation}
        For evaluation of the results the following methods were used; the \gls{MAE} and \gls{RAE} for both the uncensored data between the predicted and 'true' value was taken, the concordance index, the Brier score and a visual analysis of Grad-CAM images.

    % \vspace{-0.5cm}

\section{Results} \label{sec:results}
    \begin{table}
        % \vspace{-0.5cm}
        
        \centering
        
        \captionsetup{singlelinecheck=false, justification=centering}
        \caption{
        % \tiny
        A comparison of \gls{MAE} and \gls{RAE} calculated for the uncensored predictions and true data plus the concordance index and the Brier score.}
        
        % \vspace{-0.5cm}
        
        % \resizebox*{1.0\linewidth}{!}
        \resizebox*{1.0\linewidth}{!}
        {
            \begin{tabular}{||c|cc|c|c||}
                \hline
                                            & \textbf{\gls{MAE} UC} & \textbf{\gls{RAE} UC} & \textbf{C-Index}  & \textbf{Brier}    \\
                \hline
                \textbf{U Likelihood}       & $22.7+-1.51$          & $1.72+-0.89$          & $0.77+-0.05$      & $0.22+-0.07$      \\
                \textbf{U Likelihood CF}    & $21.5+-1.32$          & $1.98+-0.82$          & $0.80+-0.03$      & $0.18+-0.05$      \\
                \textbf{Likelihood}         & $28.9+-1.96$          & $2.23+-0.01$          & $0.76+-0.05$      & $0.25+-0.01$      \\
                \textbf{Likelihood CF}      & $25.3+-1.74$          & $2.04+-0.01$          & $0.79+-0.04$      & $0.20+-0.01$      \\
                \hline
                \textbf{Cox}                & $187 +-309 $          & $17.0+-30.7$          & $0.73+-0.04$      & $0.61+-0.28$      \\
                \textbf{Cox CF}             & $233 +-287 $          & $26.4+-21.9$          & $0.72+-0.03$      & $0.57+-0.16$      \\
                \textbf{Cox \gls{MB}}       & $?.??+-?.??$          & $?.??+-?.??$          & $?.??+-?.??$      & $?.??+-?.??$      \\
                \textbf{Cox \gls{MB} CF}    & $?.??+-?.??$          & $?.??+-?.??$          & $?.??+-?.??$      & $?.??+-?.??$      \\
                \hline
                \textbf{DeepHit}            & $38.4+-14.8$          & $3.99+-0.34$          & $0.72+-0.03$      & $0.40+-0.01$      \\
                \textbf{DeepHit CF}         & $31.3+-9.19$          & $3.50+-0.42$          & $0.71+-0.04$      & $0.41+-0.01$      \\
                \hline
            \end{tabular}
        }
        \label{tab:table}
        
        % \vspace{-0.5cm}
    \end{table}

    From~\Fref{tab:table} it can be seen that the the \gls{MAE} and \gls{RAE} are often lower for both likelihood based methods than for all other methods. The \gls{MAE} and \gls{RAE} for the DeepHit method is lower than that of the Cox based methods, while for the Cox based methods not only are their errors high but from their variance they also seem to be semi-random. The \gls{MAE} and \gls{RAE} of the uniform likelihood method is commonly lower than that of the standard likelihood method. The Brier score results also back up this assertion, however it is difficult to draw conclusions from the concordance index results. The results also seem to indicate that there is some benefit to including the clinical features, although it's uncertain if this is just because the model becomes marginally larger and in some cases it seems from the increase in \gls{RAE} that the inclusion of the clinical features may bias the results.

    % \vspace{-0.5cm}

\section{Discussion and Conclusion} \label{sec:discussion}
    From a comparison of errors and a visual analysis it appears that the likelihood based methods provide the best results most often.
    
    The model used for the DeepHit method had more parameters than the model used for all other methods (due to the output being larger), thus it may not be an entirely fair comparison. However, while using a larger model the method does not provide results significantly better than the likelihood losses.
    
    When clinical features were used it seems to improve results, although not significantly. For the increase in complexity it may not be worth including.
    
    What is not factored into the results is computation time. The Cox method without \gls{MB} is the fastest to compute, the likelohood methods are not much longer. The DeepHit method takes slightly longer than both previous methods while the cox \gls{MB} method takes magnitudes longer.

    % In the future it may be interesting to take the survival time output from the models presented here and use them as part of a generative model to predict \gls{CT} acquisitions with a lower survival time.

    
    % \vspace{-0.5cm}
    
    \AtNextBibliography{
        % \scriptsize
    }
    \printbibliography
\end{document}
